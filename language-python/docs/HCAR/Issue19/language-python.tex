\documentclass{scrreprt}
\usepackage{paralist}
\usepackage{graphicx}
\usepackage[final]{hcar}

\begin{document}

\begin{hcarentry}{language-python}
\report{Bernie Pope}
\status{stable}
%\participants{}% optional
\makeheader

Language-python is a Haskell library for lexical analysis,
parsing and pretty printing Python code. It supports versions 2.x
and 3.x of Python. The parser is implemented using the happy
parser generator, and the alex lexer generator.
It supports source accurate span information and optional parsing of
comments. A separate package called language-python-colour is available
on Hackage which demonstrates the use of the library to render
Python source in coloured XHTML.
It is also used for the syntactic analysis component of the berp
Python compiler.

\FurtherReading
The latest version of language-python, 0.3.2, is available on Hackage.
The source code repository can be browsed or downloaded from Github:
\url{http://github.com/bjpop/language-python}.

\end{hcarentry}

\end{document}
